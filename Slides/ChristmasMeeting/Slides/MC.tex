\begin{frame}
	\frametitle{MC PDFs}
	
	\li {Some PDF sets are produced using the Monte Carlo methodology, whereby a number of pseudodata replicas are generated around the nominal value. }
	\li {Central curves are constructed by taking a simple mean of all of these replicas for each mass point. }
	\li {Upper and lower uncertainties are calculated at 90\% CL and 68\% CL by excluding the appropriate number of highest and lowest replicas and then taking the maximum and minimum replica values for each mass bin. }
			
	\vspace{10pt}
	\li {NNPDF 3.0 and PDF4LHC15 both have MC PDF sets.}
	
\end{frame}





\begin{frame}
	\frametitle{MC PDFs (2)}
	
	\cleft{.48}
	\vspace{10pt}
	\li {In some cases, many replicas have negative cross sections, leading to a negative central value. }
	\lii {This is symptomatic of the absence of PDF data at high x where cross sections are driven to extremely low values.}
	\lii {Following the advice of NNPDF authors, negative replicas are set to zero.}
	\lii {For NNPDF a 100 replica set proved too small, with \textasciitilde 50\% of the replica values going negative.}
	\lii {A 1000 replica set is needed to provide at least \textasciitilde 500 positive replicas.}
	
	\cright{.52}
	\vspace{5pt}
	\begin{center}
	
	
	\includegraphics[height=\linewidth, angle=270]{plots/PDFs/NNPDF_Wcomb_allpos_posneg.pdf}
	{ \color{ATLASBlue} \scriptsize posneg = before setting negative replicas to zero.}\\
	{ \color{ATLASBlue}\scriptsize allpos = after setting negative replicas to zero.} \\
	{ \color{ATLASBlue}\scriptsize `median' = median value excluding zeros.}
	
	\end{center}
	\cend
	
\end{frame}